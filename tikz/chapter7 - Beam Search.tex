\documentclass[border=1mm]{standalone}
% \usepackage[margin=2.5cm]{geometry}

\usepackage{graphicx,tikz,tikz-layers,amsmath,ifthen,tabularray} 
\usetikzlibrary{decorations.markings,calc,positioning,arrows,shapes.geometric,arrows.meta,matrix}

\colorlet{myred}{red!80!black}
\colorlet{myblue}{blue!80!black}
\colorlet{mybluee}{myblue!80!black}
\colorlet{mygreen}{green!60!black}
\colorlet{myorange}{orange!70!red!60!black}
\colorlet{mydarkred}{red!20!black}
\colorlet{mydarkblue}{blue!40!black}
\colorlet{mydarkgreen}{green!20!black}




\begin{document}

% \resizebox{\textwidth}{!}{
\tikz[font=\small, scale=1, every node/.style={outer sep=0pt, inner sep=0pt}, w/.style={minimum width=#1}, h/.style={minimum height=#1}, s/.style={minimum size=#1}, eu/.style={shorten >=#1}, ed/.style={shorten <=#1}, line join=round]
{
\tikzset{>={Latex[length=1.5mm, width=1.25mm]}}

\node[] (start) {$<$START$>$};

% First Matrix
\matrix[matrix of nodes, right=1.5cm of start, inner sep=0pt, w=.9cm, h=.5cm,
          nodes={draw=gray!30, anchor=center, outer sep=0pt},
          row sep=-\pgflinewidth, % To remove gaps between rows
          column sep=-\pgflinewidth, % To remove gaps between columns
          nodes in empty cells % Ensure empty cells are drawn
          ] (t1)
  {
    \color{myblue} A & \color{mygreen} 0.5 \\
    \color{myblue} B &     \\
    \color{myblue} C & \color{mygreen} 0.4 \\
    \color{myblue} D &     \\
    \color{myblue} E &     \\
    \color{myblue} END &  \\
  };

\draw[myred, semithick] (t1-1-1.north west) rectangle (t1-1-2.south east);
\draw[myred, semithick] (t1-3-1.north west) rectangle (t1-3-2.south east);

% Arrows for t1
\foreach \i in {1,...,5}
\draw[->] ([xshift=0.2cm]start.east) to[out=0,in=180] (t1-\i-1.west);

% Second Matrix
\begin{scope}[xshift=5.5cm, yshift=2.75cm]
\node[draw, circle, s=.6cm] (A) {A};

\matrix[matrix of nodes, right=1.5cm of A, inner sep=0pt, w=1.2cm, h=.5cm,
          nodes={draw=gray!30, anchor=center, outer sep=0pt},
          row sep=-\pgflinewidth, % To remove gaps between rows
          column sep=-\pgflinewidth, % To remove gaps between columns
          nodes in empty cells % Ensure empty cells are drawn
          ] (t2)
  {
    \color{myblue} AA &  \\
    \color{myblue} AB &  \color{mygreen} 0.2   \\
    \color{myblue} AC &  \\
    \color{myblue} AD &     \\
    \color{myblue} AE &  \color{mygreen} 0.25   \\
    \color{myblue} A-END &  \\
  };

\draw[myred, semithick] (t2-2-1.north west) rectangle (t2-2-2.south east);
\draw[myred, semithick] (t2-5-1.north west) rectangle (t2-5-2.south east);

% Arrows for t2
\foreach \i in {1,...,5}
\draw[->] ([xshift=0.1cm]A.east) to[out=0,in=180] (t2-\i-1.west);
\end{scope}

% Third Matrix
\begin{scope}[xshift=5.6cm, yshift=-2.75cm]
\node[draw, circle, s=.6cm] (C) {C};

\matrix[matrix of nodes, right=1.5cm of C, inner sep=0pt, w=1.2cm, h=.5cm,
          nodes={draw=gray!30, anchor=center, outer sep=0pt},
          row sep=-\pgflinewidth, % To remove gaps between rows
          column sep=-\pgflinewidth, % To remove gaps between columns
          nodes in empty cells % Ensure empty cells are drawn
          ] (t3)
  {
    \color{myblue} CA &  \\
    \color{myblue} CB &     \\
    \color{myblue} CC &  \\
    \color{myblue} CD &     \\
    \color{myblue} CE &    \\
    \color{myblue} C-END &  \\
  };

% Arrows for t3
\foreach \i in {1,...,5}
\draw[->] ([xshift=0.1cm]C.east) to[out=0,in=180] (t3-\i-1.west);
\end{scope}

% Fourth Matrix
\node[draw, circle, s=.6cm, right=6cm of A, yshift=2cm] (AB) {AB};

\matrix[matrix of nodes, right=1.5cm of AB, inner sep=0pt, w=1.5cm, h=.5cm,
          nodes={draw=gray!30, anchor=center, outer sep=0pt},
          row sep=-\pgflinewidth, % To remove gaps between rows
          column sep=-\pgflinewidth, % To remove gaps between columns
          nodes in empty cells % Ensure empty cells are drawn
          ] (t4)
  {
    \color{myblue} ABA &  \\
    \color{myblue} ABB &     \\
    \color{myblue} ABC & \color{mygreen} 0.16 \\
    \color{myblue} ABD &     \\
    \color{myblue} ABE &     \\
    \color{myblue} AB-END &  \\
  };

\draw[myred, semithick] (t4-3-1.north west) rectangle (t4-3-2.south east);

% Arrows for t4
\foreach \i in {1,...,5}
\draw[->] ([xshift=0.1cm]AB.east) to[out=0,in=180] (t4-\i-1.west);

% Fifth Matrix
\node[draw, circle, s=.6cm, right=6cm of A, yshift=-2cm] (AE) {AE};

\matrix[matrix of nodes, right=1.5cm of AE, inner sep=0pt, w=1.5cm, h=.5cm,
          nodes={draw=gray!30, anchor=center, outer sep=0pt},
          row sep=-\pgflinewidth, % To remove gaps between rows
          column sep=-\pgflinewidth, % To remove gaps between columns
          nodes in empty cells % Ensure empty cells are drawn
          ] (t5)
  {
    \color{myblue} AEA &  \\
    \color{myblue} AEB &     \\
    \color{myblue} AEC &  \\
    \color{myblue} AED &  \color{mygreen} 0.2   \\
    \color{myblue} AEE &    \\
    \color{myblue} AE-END &  \\
  };

\draw[myred, semithick] (t5-4-1.north west) rectangle (t5-4-2.south east);

% Arrows for t5
\foreach \i in {1,...,5}
\draw[->] ([xshift=0.1cm]AE.east) to[out=0,in=180] (t5-\i-1.west);

% gray arrows
\tikzset{>={Latex[length=2.5mm, width=2.25mm]}}
\draw[myred, thick, ->, eu=4mm, ed=4mm] ($(t1.east)+(0,1)$) to[] (A);
\draw[myred, thick, ->, eu=4mm, ed=4mm] ($(t1.east)+(0,-1)$) to[] (C);

\draw[myred, thick, ->, eu=4mm, ed=4mm] ($(t2.east)+(0,1)$) to[] (AB);
\draw[myred, thick, ->, eu=4mm, ed=4mm] ($(t2.east)+(0,-1)$) to[] (AE);

\node[right=6cm of AB] (end1) {$<$END$>$};
\node[right=6cm of AE] (end2) {$<$END$>$};

\draw[myred, thick, ->, eu=4mm, ed=4mm] ($(t4.east)+(0,0)$) to[] ([xshift=-0.35cm]end1);
\draw[myred, thick, ->, eu=4mm, ed=4mm] ($(t5.east)+(0,0)$) to[] ([xshift=-0.35cm]end2);

% Times symbol and labels
\node[myred, scale=5] at ($(t3.east)+(-.6,0)$) {$\times$};

\node[myblue, label={[yshift=-2mm]below:\footnotesize Position 1}] (ac) at ($(t1.south)+(0,-4)$) {A, C};
\node[myblue, label={[yshift=-2mm]below:\footnotesize Position 2}] at (ac-|t3) {AB, AE};
\node[myblue, , label={[yshift=-2mm]below:\footnotesize Position 3}] at (ac-|t4) {ABC, AED};

\node[align=left, xshift=8mm] at (start|-ac) {\footnotesize Condidate Sequences};
}

% }



\end{document}
