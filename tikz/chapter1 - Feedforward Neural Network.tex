\documentclass[border=3pt,tikz]{standalone}
\usepackage{amsmath} % for aligned
\usepackage{listofitems} % for \readlist to create arrays
\usetikzlibrary{arrows.meta} % for arrow size
\usepackage[outline]{contour} % glow around text
\contourlength{1.4pt}
\usepackage{tikz-3dplot}

%%%%%
% \usepackage[cmintegrals,cmbraces]{newtxmath}
% \usepackage{ebgaramond-maths}
% \usepackage[T1]{fontenc}
\usepackage{amsmath}
\usepackage{tikz}
\usetikzlibrary{calc}

\newcommand{\tikzmarkk}[1]{\tikz[baseline,remember picture] \coordinate (#1) {};}
%%%%%

% COLORS
\usepackage{xcolor}
\colorlet{myred}{red!80!black}
\colorlet{myblue}{blue!80!black}
\colorlet{mybluee}{myblue!80!black}
\colorlet{mygreen}{green!60!black}
\colorlet{myorange}{orange!70!red!60!black}
\colorlet{mydarkred}{red!30!black}
\colorlet{mydarkblue}{blue!40!black}
\colorlet{mydarkgreen}{green!30!black}

% STYLES
\tikzset{
  >=latex, % for default LaTeX arrow head
  node/.style={thick,circle,draw=myblue,minimum size=22,inner sep=0.5,outer sep=0.6},
  node in/.style={node,green!20!black,draw=mygreen!30!black,fill=mygreen!25},
  node hidden/.style={node,blue!20!black,draw=myblue!30!black,fill=myblue!20},
  node convol/.style={node,orange!20!black,draw=myorange!30!black,fill=myorange!20},
  node out/.style={node,red!20!black,draw=myred!30!black,fill=myred!20},
  connect/.style={thick,mydarkblue}, %,line cap=round
  connect arrow/.style={-{Latex[length=4,width=3.5]},thick,mydarkblue,shorten <=0.5,shorten >=1},
  node 1/.style={node in}, % node styles, numbered for easy mapping with \nstyle
  node 2/.style={node hidden},
  node 3/.style={node out}
}
\def\nstyle{int(\lay<\Nnodlen?min(2,\lay):3)} % map layer number onto 1, 2, or 3

\begin{document}


% NEURAL NETWORK with coefficients, uniform arrows
\newcommand\setAngles[3]{
  \pgfmathanglebetweenpoints{\pgfpointanchor{#2}{center}}{\pgfpointanchor{#1}{center}}
  \pgfmathsetmacro\angmin{\pgfmathresult}
  \pgfmathanglebetweenpoints{\pgfpointanchor{#2}{center}}{\pgfpointanchor{#3}{center}}
  \pgfmathsetmacro\angmax{\pgfmathresult}
  \pgfmathsetmacro\dang{\angmax-\angmin}
  \pgfmathsetmacro\dang{\dang<0?\dang+360:\dang}
}

\begin{tikzpicture}[x=2.2cm,y=1.4cm]
  % \message{^^JNeural network with uniform arrows}
  \readlist\Nnod{4,3,3,3,2} % array of number of nodes per layer

  \foreachitem \N \in \Nnod{ % loop over layers
    \def\lay{\Ncnt} % alias of index of current layer
    \pgfmathsetmacro\prev{int(\Ncnt-1)} % number of previous layer
    \message{^^J Layer \lay, N=\N, prev=\prev ->}

    % NODES
    \foreach \i [evaluate={\y=\N/2-\i; \x=\lay; \n=\nstyle;}] in {1,...,\N}{ % loop over nodes
        \message{N\lay-\i, }
        \node[node \n] (N\lay-\i) at (\x,\y) {};
        % \node[node \n] (N\lay-\i) at (\x,\y) {$a_\i^{(\prev)}$};
      }

    % CONNECTIONS
    \foreach \i in {1,...,\N}{ % loop over nodes
        \ifnum\lay>1 % connect to previous layer
          \setAngles{N\prev-1}{N\lay-\i}{N\prev-\Nnod[\prev]} % angles in current node
          %\draw[red,thick] (N\lay-\i)++(\angmin:0.2) --++ (\angmin:-0.5) node[right,scale=0.5] {\dang};
          %\draw[blue,thick] (N\lay-\i)++(\angmax:0.2) --++ (\angmax:-0.5) node[right,scale=0.5] {\angmin, \angmax};
          \foreach \j [evaluate={\ang=\angmin+\dang*(\j-1)/(\Nnod[\prev]-1);}] %-180+(\angmax-\angmin)*\j/\Nnod[\prev]
          in {1,...,\Nnod[\prev]}{ % loop over nodes in previous layer
              \setAngles{N\lay-1}{N\prev-\j}{N\lay-\N} % angles out from previous node
              \pgfmathsetmacro\angout{\angmin+(\dang-360)*(\i-1)/(\N-1)} % number of previous layer
              %\draw[connect arrow,white,line width=1.1] (N\prev-\j.{\angout}) -- (N\lay-\i.{\ang});
              \draw[connect arrow] (N\prev-\j.{\angout}) -- (N\lay-\i.{\ang}); % connect arrows uniformly
            }
        \fi % else: nothing to connect first layer
      }
  }

  % LABELS
  \node[above=0.3,align=center,mygreen,font=\bfseries] at (N1-1.90) {Input Layer};
  \node[above=0.5,align=center,mybluee,font=\bfseries] at (N3-1.90) {Hidden Layers};
  \node[above=0.8,align=center,myred,font=\bfseries] at (N\Nnodlen-1.90) {Output Layer};

\end{tikzpicture}

\end{document}
